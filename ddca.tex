\newcommand{\Cyclus}{\textsc{Cyclus}\xspace}%
\subsection{What is it?}
\begin{frame}
  \frametitle{}
  % a comment

\begin{block}{Goals for the Demand Driven Cycamore Archetype}
\begin{enumerate}
	\item Give \Cyclus the capability to deploy supporting fuel cycle facilities to meet front-end and back-end demands of the fuel cycle. 
	\item Develop three types of prediction algorithms to achieve this capability: Non-optimizing, Deterministic Optimizing, Stochastic Optimizing. Each will improve on the previous algorithm to make the predictions more accurate. 
\end{enumerate}
\end{block}

\begin{block}{How it works}
The user inputs: 
\begin{enumerate}
	\item A specific commodity to demand for
	\item Growth rate of that specific commodity 
	\item The facilities in the simulation and their corresponding parameters
\end{enumerate}
Using the user-specified inputs, the archetype will deploy facilities and supporting facilities to meet the specified commodity's demands. It does so to ensure a continuous supply chain of materials for the production of the specified commodity to meet the growing demand. 
\end{block}


\end{frame}

\subsection{Current work}
\begin{frame}
\frametitle{}
% a comment

\begin{block}{People involved in this project}
	\textbf{University of South Carolina}\\
	PI: Dr. Anthony Scopatz \\
	Postdoc Researcher: Dr. Robert Flanagan \\
	\textbf{University of Illinois at Urbana Champaign} \\
	Co-PI: Dr. Kathryn Huff \\
	Graduate Student Researchers: Gwendolyn Chee and Jin Whan Bae (Teddy)
\end{block}

\begin{block}{Current work being done by UIUC}
	\begin{enumerate}
		\item Numerical experiments to test the Non-optimizing and deterministic optimizing prediction algorithms. 
		\item Reports for each type of prediction algorithm. The report for the non-optimizing case can be found \cite{bae_numerical_2018}
	\end{enumerate}
\end{block}

\begin{block}{When will it be ready? }
	Non-Optimizing Algorithm: August 2018	\\
	Deterministic-Optimizing Algorithm: October 2018	
\end{block}


\end{frame}